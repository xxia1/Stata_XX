\section{Apps and Project Management} 
In this section, I go through the Apps needed and introduce some ways to organize folders and files for each project, so they can be accessed more efficiently. 
\subsection{Stata}
Stata is not free, so you need a license. Duke students can purchase an annual license here: \url{https://software.duke.edu/products/statanowse}. The installation process is very straightforward. Just follow the instructions after you purchase the license. 
\subsection{\LaTeX}
{\LaTeX} is a text editor (i.e., an app you use to write stuff), basically Microsoft Word on steroids. It can link exhibits to other files and automatically update and format them. It can also take care of references to section numbers, exhibit numbers, literature, etc., every time they change, which can be a real pain in Word. This article itself is procured by \LaTeX.

There are different ways to run \LaTeX. Here I introduce two ways to do it, depending on whether you use Dropbox (since Dropbox is currently not free for Duke students).

\paragraph{If you use Dropbox}
The easiest way is to use Overleaf, since it doesn't require installing any software and is smarter in autofill and spellcheck. In order to link the {\LaTeX} script on Overleaf to the exhibits your code exports, you need a premium subscription to Overleaf. Luckily, it's free for Duke students. Just go to \url{https://www.overleaf.com/edu/duke} and register using your Duke email. If you previously had an Overleaf account, it also allows you to combine two accounts (i.e., link the Duke email to your old account so the premium functions are free). 

Once you have the premium subscription, go to 
\begin{verbatim}
    Account Settings/Project synchronisation/Dropbox Sync
\end{verbatim}
to link your Overleaf to your Dropbox account. Then, in your Dropbox, under \verb|Apps/Overleaf|, you will find folders for each project you create on Overleaf, and vice versa. 

\paragraph{If you don't use Dropbox, or your project is on a server, etc., that doesn't have an internet connection}
The most flexible way is to install {\LaTeX} locally on your computer. The installation comes in two parts. First, you need to install the {\LaTeX} itself, which comes with all kinds of packages and codes that are required to run {\LaTeX}. For Mac, one can install MacTeX; for Windows, one can install MiKTeX. Second, you need a text editor for {\LaTeX}. The installation of {\LaTeX} often comes with a default editor that is very simple and doesn't support many functions, such as autofill and spellcheck. Two editors that I used are TeXstudio (Mac only) and VS Code. TeXstudio is more robust, but the spellcheck function is very basic. VS Code requires adding some extensions and has fancier functions, but it's less robust; sometimes it reports strange errors that you don't see on Overleaf or TeXstudio. Just try them out and see which you like the most. 

\subsection{Folder Structure}
Different people have different ways to organize their project folders. No matter how you do it, the setup should at least meet two criteria:
\begin{enumerate}
    \item You know where to find your stuff;
    \item The code and {\LaTeX} script should be able to easily reference the paths of different files.
\end{enumerate}
For each project, I recommend separating its files into three big folders: 
\begin{itemize}
    \item \verb|/Data|: contains \verb|/raw| for raw data files that you never edit, and \verb|/derived| for data processed or generated by your codes. 
    \item \verb|/Codes|: contains your codes for analysis. They should be clearly named and indexed (e.g., \verb|01_...|, \verb|02_...|). In these codes, cleaned data sets are saved to \verb|/Data/derived|, and exhibits are saved in \verb|/Output|
    \item \verb|/Output|: contains \verb|/Tables| and \verb|/Figures| for exhibits produced by your codes. If you use Overleaf, this folder should be at \verb|/Dropbox/Apps/Overleaf/[Project Name]/|; if you use {\LaTeX} locally, this can be saved in the same place as the two folders above. 
\end{itemize}
There is also your {\LaTeX} script, in which the exhibits are linked to the path in \verb|\Output|. If you run {\LaTeX} locally, you may have the script saved in \verb|/Output/Tex/|; otherwise, the {\LaTeX} script should be in \verb|/Dropbox/Apps/Overleaf/[Project Name]/| alongside \verb|/Output|. 

Depending on the project size, one can also break things down into different tasks, with each task having its own set of code, exhibits, derived data files, and even {\LaTeX} report. I'd recommend trying different structures to find the one that works for you and your project. 

\begin{figure}[H]
    \centering
    \caption{Folder Structure}
    \begin{tikzpicture}[
        node distance=2.8cm and 1.5cm,
        box/.style={rectangle, minimum width=2.4cm, minimum height=1.8cm, text centered,  fill=cyan!70, font=\bfseries\color{white}},
        doc/.style={rectangle, draw=gray, fill=gray, minimum width=1.2cm, minimum height=1.8cm},
        arrow/.style={-{Stealth}, thick}
    ]
    
    % Nodes
    \node[box] (data) {Data};
    \node[box, right=2.7cm of data] (codes) {Codes};
    \node[box, right=2.7cm of codes] (output) {Output};
    \node[doc, right=2.7cm of output] (latex) {\color{white}\shortstack{\LaTeX \\Script}};
    % \node[above=0.1cm of latex] {LaTeX Script};
    
    % % Document lines
    % \draw[white, line width=0.5pt] (latex.south west) ++(0.2,0.2) -- ++(0.8,0);
    % \draw[white, line width=0.5pt] (latex.south west) ++(0.2,0.45) -- ++(0.8,0);
    % \draw[white, line width=0.5pt] (latex.south west) ++(0.2,0.7) -- ++(0.8,0);
    
    % Arrows
    \draw[arrow, bend left=30] (data) to node[above] {Read Raw/Derived Data} (codes);
    \draw[arrow, bend right=-30] (codes) to node[below] {Save Derived Data} (data);
    \draw[arrow] (codes) -- (output) node[midway, below] {Save Exhibits};
    \draw[arrow] (latex) -- (output) node[midway, below] {Read Exhibits};
    \end{tikzpicture}
    
    % \label{fig:enter-label}
\end{figure}

In the companion repository, you can find how this project is organized. 

\subsection{A Project Diary}
Believe it or not, no matter how well-organized the folders are, one can still easily lose track of things, especially \textit{which file} is produced by \textit{what} and saved \textit{where}. This is why keeping a project diary can be very helpful, especially for projects you think you would not touch for a few days --- never underestimate how much and how fast you can forget.  

A project diary documents exactly this: \textit{which file} is produced by \textit{what} and saved \textit{where}. You may organize by code: for each code script, what it does, what it produces, and where the output is saved. You may also organize the document by output: for each output, which script produces it, and where it's saved.

One key is to keep it simple. Just note down the most important stuff. If it's too detailed or complex, you'd feel too lazy to update it promptly. Once it's not up to date, it's useless, since the next time you get back to it, you will forget what has been updated and what has not. 

You may do it anywhere you want. I like using the Wiki page on GitHub, since it's free, easy to access, and can track updates you make. One can use Notion, too, but I personally feel it's overkill. It loads slowly, often asks me to log in again, and has too many functions that we don't need for this. Again, keep it simple, just so you can keep updating it. I think key functions we need on this are:
\begin{itemize}
    \item Easy access; quick to open;
    \item Tracks historical edits;
    \item Easy to type file and folder names and easy to search. For example, in the Git Wiki page, you can just type stuff between \verb|``| and the text inside will look different from normal text.
\end{itemize}

